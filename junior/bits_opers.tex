\section{Битовый бой}

\subsection{Двоичная система счисления}

\begin{definition}[Системы счисления]
    \[
        \overline{a_{n - 1}\ldots a_1 a_0}_x \vcentcolon = \sum\limits_{i = 0}^{n - 1} a_i\cdot x^i
    \]
\end{definition}

В компьютере числа хранятся в двоичной системе счисления, причём хранение отрицательных чисел производится с помощью, так называемого, \textit{обратного кода}. Первый бит отвечает за знак числа (для неотрицательных он равен $0$, для отрицательных $1$). Пусть имеем двоичное число  $A = 1101\underbrace{0\ldots0}_{28\, \text{нулей}}$. Как найти обратное к нему? Чтобы арифметика работала, нужно, чтобы такое число $B$ при сложении с $A$ давало $0$. Этого можно добиться переполнением, то есть, нам нужно получить число $1\underbrace{0\ldots0}_{32\, \text{нуля}}$, что~в типе \texttt{int} является нулём (ведь он хранит только 32 младших бита). Так, нам подойдёт число $B = 0011\underbrace{0\ldots0}_{28\,\text{нулей}}$.

\subsection{Битовые операции}

\begin{definition}
    \begin{enumerate}[nolistsep]
        \item \textit{Конъюнкция} --- побитовое <<и>> ($x \wedge y$);
        \item \textit{Дизъюнкция} --- побитовое <<или>> ($x \vee y$);
        \item \textit{XOR} --- сложение $\bmod\;2$ ($x \oplus y$);
        \item \textit{Инверсия} --- побитовое отрицание ($\sim x$);
        \item \textit{Побитовый сдвиг вправо} --- умножение на $2^k$ ($x\;\texttt{<<}\;k$);
        \item \textit{Побитовый сдвиг влево} --- целочисленное деление на $2^k$ ($x\;\texttt{>>}\;k$).
    \end{enumerate}
\end{definition}

\problem{Определить значение $i$-го бита числа $x$}

\solution{$x \wedge (1\;\texttt{<<}\;i)$}

\problem{Инвертировать $i$-й бит числа $x$}

\solution{$x \oplus (1\;\texttt{<<}\;i)$}

